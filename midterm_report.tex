%
% 公立はこだて未来大学卒業研究中間報告書[情報システム/高度ICTコース]
%
%         ファイル名:"midterm_report.tex"
%
\documentstyle[funinfosys,graphicx,url,11pt]{jarticle}
\author{% 
b1014120 永井陽太\\指導教員 : 松原克弥
}

\title{教室PCの余剰資源を活用した学内向けオープンクラウドの開発}
\etitle{Development of open cloud service utilized surplus resources of computers in computer room on campus}
\eauthor{Yota Nagai}
\abstract{和文は300から400文字で記述すること.}
\keywords{クラウド, 仮想化}
\eabstract{English should be written in 100 to 150 words.}
\ekeywords{Cloud, Virtualization}
\begin{document}
\maketitle
%\vspace*{-.5cm}

\section{はじめに}
\subsection{背景}
ITビジネスにおいて,クラウドサービスの利用はもはや一般的となってきている.教育機関においても,クラウドサービスの利用が広がりつつある.
教育期間においては,BYODの普及により,教室PCの利用の機会が減少している.
\subsection{提案システム}
本稿では,教室PCの余剰資源を活用した学内向けオープンクラウドの実現方式を検討する.

\section{教育機関におけるクラウド利用}

\subsection{パブリッククラウドの利用例}
\subsection{プライベートクラウドの利用例}
\subsection{大学の既存環境と問題点}

\section{提案するクラウドサービスの開発について}
\subsection{技術的課題}
\subsubsection{不安定な計算資源}
\subsection{実現手法}
\subsubsection{不規則な計算ノードの変更に柔軟に対応}
\section{実装状況}
\subsection{研究室におけるクラウドの構築}
\subsection{OpenStack}

\section{おわりに}
\subsection{まとめ}
\subsection{今後の課題}

\begin{thebibliography}{99}
\bibitem{marumaru}
	○○△△, システム情報科学会論文誌, 2, 13-19, 2002.
\bibitem{abc}
	A.B.Cdddddd, J. Systems Information Science, 11, 1145-1159, 2001.
\bibitem{batubatu}
	○○××, □□△△, システム情報科学, ☆☆出版, 1999, 20-21.
\bibitem{efghij}
	E.Fggg and H.Ijjj, Electrical Engineering, KKPress, 2003, 281-284.
\end{thebibliography}
\end{document}
%
%
% EOF 
