%
% 公立はこだて未来大学卒業研究中間報告書[情報システム/高度ICTコース]
%
%         ファイル名:"midterm_report.tex"
%
\documentstyle[funinfosys,graphicx,url,11pt]{jarticle}
\author{% 
b1014120 永井陽太\\指導教員 : 松原克弥
}

\title{教室PCの余剰資源を活用した学内向けオープンクラウドの開発}
\etitle{Development of open cloud service utilized surplus resources of computers in computer room on campus}
\eauthor{Yota Nagai}
\abstract{和文は300から400文字で記述すること.}
\keywords{クラウド, 仮想化}
\eabstract{English should be written in 100 to 150 words.}
\ekeywords{Cloud, Virtualization}
\begin{document}
\maketitle
%\vspace*{-.5cm}

\section{はじめに}
\subsection{背景}
近年,クラウドコンピューティングはITビジネスにおいて中核を担う技術となり,当たり前に利用されるようになってきている.
その広がりは教育機関においても同様である.
近年のICT教育の現場では,実践的な演習活動を通して経験を培うようになってきており,
クライアントサーバーモデルのシステムを構築したり,他のWebサービスと連携を行うシステムを構築する上でも,
学生1人に対して複数台のマシンが利用できる環境を提供することが教育期間に求められる様になってきている.
\par 教育機関がクラウドサービスを学生に提供する方法としては,
\begin{enumerate}
	\item 商用のパブリッククラウドサービスを利用する方法
	\item 学内にプライベートクラウドを構築する方法
\end{enumerate}
という2つの方法が存在するが,
パブリッククラウドサービスを利用する場合は,サービスの利用料に応じて料金が発生してしまい,
学生全体にクラウドサービスを提供することを考えると膨大なコストがかかってしまう.
また,学内にプライベートクラウドを構築する場合は,クラウドシステムをデプロイするためには
ハイスペックなサーバー機が必要となり,多大な初期投資が必要になってしまう.
\par 一方,ICT教育期間においてはBYODが普及してきており,学生1人1人が個人でマシンを持つことが当たり前になってきており
演習環境として用意されているPC教室のPCがあまり利用されいないという現状がある.
つまり,PC教室の計算資源が余っており,活用できていない.
\subsection{提案システム}
本稿では,教室PCの余剰資源を活用した学内向けオープンクラウドの実現方式を検討する.
本検討では,大学内の教室PCを計算資源の対象とすることで,技術的課題の洗い出しと有効性を確かめる.
\par 本提案では,学生による教室PCの利用や講義による教室PCの利用によって,計算資源の激しい変化に柔軟に対応する必要がある.
そこで,計算機として利用できる教室PCを動的に管理するための仕組みと,
仮想インスタンスを立ち上げる際に,最も負荷が少ない計算機を選択する仕組みと,
負荷が高くなった計算機から負荷が少ない計算機へ仮想インスタンスをライブマイグレーションする仕組みを実現する.
\par 以降,第2章では,現在の大学教育期間におけるクラウドサービス利用の現状について述べる.
第3章で,教室PCを利用した学内オープンクラウドを実現するための技術的課題を示し,その解決方法について論じる.
第4章では,実装状況について述べる.第5章では,まとめと今後の課題について述べる.

\section{教育機関におけるクラウド利用}
\subsection{パブリッククラウドの利用例}
\subsection{プライベートクラウドの利用例}
\subsection{大学の既存環境と問題点}

\section{提案するクラウドサービスの開発について}
\subsection{技術的課題}
\subsubsection{不安定な計算資源}
\subsection{実現手法}
\subsubsection{不規則な計算ノードの変更に柔軟に対応}

\section{実装状況}
\subsection{研究室におけるクラウドの構築}
\subsection{OpenStack}

\section{おわりに}
\subsection{まとめ}
\subsection{今後の課題}

\begin{thebibliography}{99}
\bibitem{marumaru}
	○○△△, システム情報科学会論文誌, 2, 13-19, 2002.
\bibitem{abc}
	A.B.Cdddddd, J. Systems Information Science, 11, 1145-1159, 2001.
\bibitem{batubatu}
	○○××, □□△△, システム情報科学, ☆☆出版, 1999, 20-21.
\bibitem{efghij}
	E.Fggg and H.Ijjj, Electrical Engineering, KKPress, 2003, 281-284.
\end{thebibliography}
\end{document}
%
%
% EOF 
