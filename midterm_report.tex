%
% 公立はこだて未来大学卒業研究中間報告書[情報システム/高度ICTコース]
%
%         ファイル名:"midterm_report.tex"
%
\documentstyle[funinfosys,graphicx,url,11pt]{jarticle}
\author{% 
b1014120 永井陽太\\指導教員 : 松原克弥
}

\title{教室PCの余剰資源を活用した学内向けオープンクラウドの開発}
\etitle{Development of open cloud service utilized surplus resources of computers in computer room on campus}
\eauthor{Yota Nagai}
\abstract{和文は300から400文字で記述すること.}
\keywords{クラウド, 仮想化}
\eabstract{English should be written in 100 to 150 words.}
\ekeywords{Cloud, Virtualization}
\begin{document}
\maketitle
%\vspace*{-.5cm}

\section{はじめに}
\subsection{背景}
近年,クラウドコンピューティングはITビジネスにおいて中核を担う技術となり,当たり前に利用されるようになってきている.
その広がりは教育機関においても同様である.
近年のICT教育の現場では,実践的な演習活動を通して経験を培うようになってきており,
クライアントサーバーモデルのシステムを構築したり,他のWebサービスと連携を行うシステムを構築する上でも,
学生1人に対して複数台のマシンが利用できる環境を提供することが教育期間に求められる様になってきている.
\par 教育機関がクラウドサービスを学生に提供する方法としては,
\begin{enumerate}
	\item 商用のパブリッククラウドサービスを利用する方法
	\item 学内にプライベートクラウドを構築する方法
\end{enumerate}
という2つの方法が存在するが,
パブリッククラウドサービスを利用する場合は,サービスの利用料に応じて料金が発生してしまい,
学生全体にクラウドサービスを提供することを考えると膨大なコストがかかってしまう.
また,学内にプライベートクラウドを構築する場合は,クラウドシステムをデプロイするためには
ハイスペックなサーバー機が必要となり,多大な初期投資が必要になってしまう.
\par 一方,ICT教育期間においてはBYODが普及してきており,学生1人1人が個人でマシンを持つことが当たり前になってきており
演習環境として用意されているPC教室のPCがあまり利用されいないという現状がある.
つまり,PC教室の計算資源が余っており,活用できていない.
\subsection{提案システム}
本稿では,教室PCの余剰資源を活用した学内向けオープンクラウドの実現方式を検討する.
本検討では,大学内の教室PCを計算資源の対象とすることで,技術的課題の洗い出しと有効性を確かめる.
\par 本提案では,学生による教室PCの利用や講義による教室PCの利用によって,計算資源の激しい変化に柔軟に対応する必要がある.
そこで,計算機として利用できる教室PCを動的に管理するための仕組みと,
仮想インスタンスを立ち上げる際に,最も負荷が少ない計算機を選択する仕組みと,
負荷が高くなった計算機から負荷が少ない計算機へ仮想インスタンスをライブマイグレーションする仕組みを実現する.
\par 以降,第2章では,現在の大学教育期間におけるクラウドサービス利用の現状について述べる.
第3章で,教室PCを利用した学内オープンクラウドを実現するための技術的課題を示し,その解決方法について論じる.
第4章では,実装状況について述べる.第5章では,まとめと今後の課題について述べる.

\section{教育機関におけるクラウド利用の利点・欠点および現状}
\subsection{パブリッククラウドの利点と欠点}
パブリッククラウドとはGoogle Cloud Platform,Amazon Web Service,さくらのクラウド等の商用のクラウドサービスのことである.
これらのパブリッククラウドを利用する利点としては,
\begin{enumerate}
	\item クラウドサービスを利用することにより,高額なハードウェアを購入せずにサーバを立ち上げることが可能になり,初期投資が圧縮出来る点
	\item サーバの消費電力コスト,サーバ室の冷却コスト,無停電電源装置の購入および維持等の維持管理コストが削減出来る点
\end{enumerate}
以上の2点が挙げられる.
\\ 欠点としては,
\begin{enumerate}
	\item クラウド上にサーバを立ち上げることにより,サーバが世界中に公開されて,不正アクセスなどによるサイバー攻撃の危険に晒されるという情報セキュリティのリスク
	\item 学内ネットワークに障害が発生し,外部のネットワークに繋がらなくなってしまった場合などに,クラウド上のサーバにアクセスできなくなるというネットワーク切断のリスク
\end{enumerate}
以上の2点が挙げられる.
\subsection{プライベートクラウドの利点と欠点}
プライベートクラウドとは組織内の独自の環境で構築・運用されているクラウドシステムのことである.このようなプライベートクラウドを利用する利点としては,
\begin{enumerate}
	\item 独自の環境にオンプレミスに構築・運用されていることによってクローズドなシステムになるので高いセキュリティを実現することが出来る点
	\item すでに組織内で運用されている様々なサービスとの連携が可能になり,組織内のニーズに合わせて柔軟にシステムを構成することが出来る点
\end{enumerate}
以上2点が挙げられる.\\
プライベートクラウドを利用する際の欠点としては,
\begin{enumerate}
	\item プライベートクラウド構築の際に,ハイスペックなサーバ機が必要になったり,サーバを設置する環境を整えたりする等の多大な初期投資が必要になる点
	\item 組織内に設置したサーバに障害が発生した場合に備えて,障害に対処するサポート体制を整備するための人的なコストがかかってしまう点
\end{enumerate}
以上の2点が挙げられる.

\subsection{大学の既存環境と問題点}
現在のICT教育機関では学生がPBL(Project Based Learning)を通して,実践的な経験を培うことで学生の技術力を向上させることが主流になってきている.
実践的な演習活動を行っていく上で,例えば,クライアントサーバモデルのシステムを構築したり,
IoTデバイスでセンシングしたデータを集約するためのサーバを構築したり,他のWebアプリケーションをネットワークを通して連携するシステムを構築したりする様になってきている.
これらのシステム構成を実現するためには学生1人に対して1台の端末が提供される環境では不十分になってきている.
従って,近年のICT教育機関には学生1人に対して複数台のマシンを提供できる環境を整備することが求められている.
\par 一方,現在のICT教育環境では,BYOD(Bring Your Own Device)が普及しており,学生1人1人が個人で専用のマシンを有しているという現状がある.
BYODの普及により,現在は演習用の端末を配置した「PC教室」の利用が減ってきており,メンテナンスのコストがかかる「PC教室」自体が不要になってきているのではないかという主張がある.
しかし,BYODが普及していることによって学生が持参してくるPCが多様化し,学生がどのようなPCを持ってくるかわからない状況になってきている.
この様な状況下では,「統一の演習環境」は必ず必要になるので「PC教室」を撤廃することは難しい.
つまり,「PC教室」には余剰な計算資源が存在しており,活用されていないという問題がある.

\section{提案するクラウドサービスについて}
\subsection{技術的課題}
\subsubsection{不安定な計算資源}
\subsection{実現手法}
\subsubsection{不規則な計算ノードの変更に柔軟に対応}

\section{実装状況}
\subsection{研究室におけるクラウドの構築}
\subsection{OpenStack}

\section{おわりに}
\subsection{まとめ}
\subsection{今後の課題}

\begin{thebibliography}{99}
\bibitem{marumaru}
	○○△△, システム情報科学会論文誌, 2, 13-19, 2002.
\bibitem{abc}
	A.B.Cdddddd, J. Systems Information Science, 11, 1145-1159, 2001.
\bibitem{batubatu}
	○○××, □□△△, システム情報科学, ☆☆出版, 1999, 20-21.
\bibitem{efghij}
	E.Fggg and H.Ijjj, Electrical Engineering, KKPress, 2003, 281-284.
\end{thebibliography}
\end{document}
%
%
% EOF 
