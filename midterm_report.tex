%
% 公立はこだて未来大学卒業研究中間報告書[情報システム/高度ICTコース]
%
%         ファイル名:"midterm_report.tex"
%
\documentclass[11pt,a4paper]{jsarticle}
\usepackage{funinfosys}
\usepackage{graphicx}
\usepackage{url}

\author{% 
b1014120 永井陽太\\指導教員 : 松原克弥
}
\course{Advanced ICT Course} %% 高度ICTコースの場合はこちらを使用
\title{教室PCの余剰資源を活用した学内向けオープンクラウドの構築}
\etitle{Development of open cloud service utilized surplus resources of computers in computer room on campus}
\eauthor{Yota Nagai}
\abstract{和文は300から400文字で記述すること.}
\keywords{クラウド, 仮想化}
\eabstract{English should be written in 100 to 150 words.}
\ekeywords{Cloud, Virtualization}
\begin{document}
\maketitle
%\vspace*{-.5cm}

\section{はじめに}

\subsection{背景}
近年,クラウドコンピューティング技術ははITビジネスにおいて,必要不可欠な技術となってきている.
また,クラウドコンピューティング技術は教育機関にも普及してきている.
近年のICT教育の現場では,実践的な演習活動を行うようになってきており,
クライアントサーバモデルのシステムを構築したり,他のWebサービスと連携を行うシステムを構築する上でも,
学生1人に対して複数台のマシンが利用できる環境を提供することが教育機関に求められてきている.
\par 教育機関がクラウドサービスを学生に提供する方法としては,
\begin{enumerate}
	\item 商用のクラウドサービスを利用する
	\item 学内にプライベートクラウドを構築する
\end{enumerate}
という2つの方法が存在するが,
商用のクラウドサービスを利用する場合は,サービスの利用料に応じて料金が発生してしまい,
学生全体にクラウドを展開することを考えると膨大なコストがかかってしまう.
また,学内にプライベートクラウドを構築する場合は,高価なサーバ機が必要となり,多大な初期投資が必要である.
\par 一方,ICT教育機関においてはBYOD,BYOPCが普及してきており,学生1人1人が個人でマシンを持つことが当たり前になってきており
演習環境として用意されているPC教室のPCがあまり利用されいないという現状がある.
つまり,PC教室の計算資源が余っており,活用できていない.
\par また,近年はオープンクラウドと呼ばれる,オープンソースベースのクラウド基盤が普及してきており
クラウドの構築が以前よりも比較的容易になってきている.

\subsection{提案システム}
本稿では,教室PCの余剰資源を活用した学内向けオープンクラウドの実現方式を検討する.
本検討では,大学内の教室PCを計算資源の対象とすることで,技術的課題の洗い出しと有効性を確かめる.
\par 本提案では,学生による教室PCの利用や講義による教室PCの利用によって,計算資源の激しい変化に柔軟に対応する必要がある.
そこで,計算機として利用できる教室PCを動的に管理するための仕組みと,
仮想インスタンスを立ち上げる際に,最も負荷が少ない計算機を選択する仕組みと,
負荷が高くなった計算機から負荷が少ない計算機へ仮想インスタンスをライブマイグレーションする仕組みを実現する.
\par 本提案が実現することで,以下に述べるような効果を期待できる.
\begin{itemize}
	\item 学内の余剰計算機を利用することで,クラウド構築の際の初期投資を抑えながら,学生に対して1人複数台のマシンを提供できる環境を構築することができる.
\end{itemize}

\par 以降,第2章では,現在の大学教育機関におけるクラウドサービス利用の現状について述べる.
第3章で,教室PCを利用した学内オープンクラウドを実現するための技術的課題を示し,その解決方法について論じる.
第4章では,実装状況について述べる.第5章では,まとめと今後の課題について述べる.

\section{教育機関におけるクラウド利用の利点・欠点および現状}
\subsection{パブリッククラウドの利点と欠点}
パブリッククラウドとはGoogle Cloud Platform,Amazon Web Service,さくらのクラウド等の商用のクラウドサービスのことである.
これらのパブリッククラウドを利用する利点としては,
\begin{enumerate}
	\item クラウドサービスを利用することにより,高額なハードウェアを購入せずにサーバを立ち上げることが可能になり,初期投資が圧縮出来る点
	\item サーバの消費電力コスト,サーバ室の冷却コスト,無停電電源装置の購入および維持等の維持管理コストが削減出来る点
\end{enumerate}
以上の2点が挙げられる.
\\ 欠点としては,
\begin{enumerate}
	\item クラウド上にサーバを立ち上げることにより,サーバが世界中に公開されて,不正アクセスなどによるサイバー攻撃の危険に晒されるという情報セキュリティのリスク
	\item 学内ネットワークに障害が発生し,外部のネットワークに繋がらなくなってしまった場合などに,クラウド上のサーバにアクセスできなくなるというネットワーク切断のリスク
\end{enumerate}
以上の2点が挙げられる.
\subsection{プライベートクラウドの利点と欠点}
プライベートクラウドとは組織内の独自の環境で構築・運用されているクラウドシステムのことである.このようなプライベートクラウドを利用する利点としては,
\begin{enumerate}
	\item 独自の環境にオンプレミスに構築・運用されていることによってクローズドなシステムになるので高いセキュリティを実現することが出来る点
	\item すでに組織内で運用されている様々なサービスとの連携が可能になり,組織内のニーズに合わせて柔軟にシステムを構成することが出来る点
\end{enumerate}
以上2点が挙げられる.\\
プライベートクラウドを利用する際の欠点としては,
\begin{enumerate}
	\item プライベートクラウド構築の際に,ハイスペックなサーバ機が必要になったり,サーバを設置する環境を整えたりする等の多大な初期投資が必要になる点
	\item 組織内に設置したサーバに障害が発生した場合に備えて,障害に対処するサポート体制を整備するための人的なコストがかかってしまう点
\end{enumerate}
以上の2点が挙げられる.

\subsection{大学の既存環境と問題点}
現在のICT教育機関では学生がPBL(Project Based Learning)を通して,実践的な経験を培うことで学生の技術力を向上させることが主流になってきている.
実践的な演習活動を行っていく上で,例えば,クライアントサーバモデルのシステムを構築したり,
IoTデバイスでセンシングしたデータを集約するためのサーバを構築したり,他のWebアプリケーションをネットワークを通して連携するシステムを構築したりする様になってきている.
これらのシステム構成を実現するためには学生1人に対して1台の端末が提供される環境では不十分になってきている.
従って,近年のICT教育機関には学生1人に対して複数台のマシンを提供できる環境を整備することが求められている.
\par 一方,現在のICT教育環境では,BYOD(Bring Your Own Device)が普及しており,学生1人1人が個人で専用のマシンを有しているという現状がある.
BYODの普及により,現在は演習用の端末を配置した「PC教室」の利用が減ってきており,メンテナンスのコストがかかる「PC教室」自体が不要になってきているのではないかという主張がある.
しかし,BYODが普及していることによって学生が持参してくるPCが多様化し,学生がどのようなPCを持ってくるかわからない状況になってきている.
この様な状況下では,「統一の演習環境」は必ず必要になるので「PC教室」を撤廃することは難しい.
つまり,「PC教室」には余剰な計算資源が存在しており,活用されていないという問題がある.

\section{提案するクラウドサービスについて}
PC教室の計算資源を活用し,学内にクラウドサービスを構築するためには,激しく変化する計算資源に対して柔軟に対応しなければならない.
本章では,教室PCの余剰資源を活用したクラウドサービスを実現するにあたっての課題と解決方法について述べる.
また,本検討の対象OSはLinuxとMacOS Xとする.
\subsection{技術的課題}
\subsubsection{不安定な計算資源}
教室PCは学内にいる任意の人の手に触れる状況にある.つまり,学内の全ての人間が教室PCを利用でき,最悪の場合,教室PCの電源を切ることが出来る状況にある.
また,プログラミング演習の講義が行われる場合,半数以上の教室PCが学生によって利用されてしまうので,学内クラウドの為の計算資源が枯渇してしまう.
この様な状況下で,高い可用性を実現しながらクラウドサービスを提供することはとても困難である.
\subsection{実現手法}
\subsubsection{不安定な計算資源に柔軟に対応}
このような不安定な計算資源を元にクラウドサービスを構築する場合でも,以下の4つの機構を用意することで計算資源の変化に柔軟に対応し,ベストエフォートを実現する.
\begin{itemize}
	\item 教室PCの死活管理をする機構
	\item 仮想インスタンスを立ち上げる際に,最も負荷の少ない計算機を選択する機構
	\item 仮想インスタンスが動作している計算機の負荷が高くなった場合,不可の少ない計算機にライブマイグレーションする機構
	\item 教室PCが突然使えなくなってしまう場合を考えて,定期的に仮想インスタンスのバックアップを作成する機構
\end{itemize}

\section{実装状況}
\subsection{研究室におけるクラウドの構築}
実際に教室PCを利用してクラウドを構築するためには,PC教室を管理している組織にPC教室の利用を申請する必要がある.
そのためにはPC教室を管理している組織を説得するための実績が必要になる.
そこで,現在は研究室内にOpenStackというOSSを利用してプライベートなクラウドサービスを構築している段階である.
\subsection{OpenStack}
OpenStackとは,大量の計算機,ストレージ,ネットワーク資源をコントロールし,パブリックもしくはプライベートなクラウドを構築するためのオープンソースソフトウェアである.
OpenStackはマイクロサービスアーキテクチャを採用しており,6つのコアサービスの連携によってクラウドコンピューティングを実現している.
\subsubsection{KeyStone}
OpenStack参加しているサービスのエンドポイントの管理と,認証の管理を担っている.
エンドユーザはKeyStoneから認証された時にOpenStack内の他のサービスのエンドポイントとアクセストークンを取得する.
\subsubsection{Nova}
仮想インスタンスを立ち上げる為の計算ノードの管理及び仮想インスタンスの管理を担っている.
計算ノードの情報を,随時コントローラノードに送信することで,計算ノードの死活管理や,仮想インスタンスを立ち上げる最適な計算ノードを決定する役割も担っている.
\subsubsection{Glace}
仮想インスタンスを立ち上げる際のイメージの提供及びイメージのメタ情報の管理を担っている.
実際のイメージは後述するSwiftなどのオブジェクトストレージ,もしくはCinder等のブロックストレージに保存している.
\subsubsection{Swift}
イメージファイルやメディアファイル等のオブジェクトを保存する為のストレージの提供及び管理を担っている.
レプリケーションにも対応しており,障害への対策も可能にする.
\subsubsection{Cinder}
仮想インスタンスへのブロックストレージの提供及び管理を担っている.
\subsubsection{Neutron}
OpenStack内のネットワークリソースの管理及び仮想インスタンスへのネットワークリソースの提供を担っている.

\section{おわりに}
\subsection{まとめ}
本稿では,教室PCの余剰資源を活用した学内向けオープンクラウドの実現方式の提案を行った.
対象OSをLinuxとMacOS Xとし,教室PCの不安定な計算資源に対して,
計算ノードの死活管理,最適な計算ノードの選択,
高負荷な計算ノードから低負荷な計算ノードへのライブマイグレーション及び
定期的な仮想インスタンスのバックアップによってベストエフォートなクラウドサービスの構築の検討を行った.
提案するシステムがもたらす効果は,
学内の余剰計算機を用いることでプライベートクラウド構築の欠点である,
初期投資を抑えながら,学生に対して1人複数台のマシンを提供する環境の構築が可能になることが見込める.
\subsection{今後の課題}
今後は,学生1人に対して具体的に何台マシンを提供できるようにするのか決定する必要がある.
また,どのように学内の認証機構との連携を行うのか,考慮する必要がある.
さらに,学生が仮想インスタンスを操作するためのWebUIをOpenStackのデフォルトの状態から,学生が使いやすいUIを検討する必要がある.

\begin{thebibliography}{99}
\bibitem{marumaru}
	○○△△, システム情報科学会論文誌, 2, 13-19, 2002.
\bibitem{abc}
	A.B.Cdddddd, J. Systems Information Science, 11, 1145-1159, 2001.
\bibitem{batubatu}
	○○××, □□△△, システム情報科学, ☆☆出版, 1999, 20-21.
\bibitem{efghij}
	E.Fggg and H.Ijjj, Electrical Engineering, KKPress, 2003, 281-284.
\end{thebibliography}
\end{document}
%
%
% EOF 
